% UCL Thesis LaTeX Template
%  (c) Ian Kirker, 2014
% 
% This is a template/skeleton for PhD/MPhil/MRes theses.
%
% It uses a rather split-up file structure because this tends to
%  work well for large, complex documents.
% We suggest using one file per chapter, but you may wish to use more
%  or fewer separate files than that.
% We've also separated out various bits of configuration into their
%  own files, to keep everything neat.
% Note that the \input command just streams in whatever file you give
%  it, while the \include command adds a page break, and does some
%  extra organisation to make compilation faster. Note that you can't
%  use \include inside an \include-d file.
% We suggest using \input for settings and configuration files that
%  you always want to use, and \include for each section of content.
% If you do that, it also means you can use the \includeonly statement
%  to only compile up the section you're currently interested in.
% You might also want to put figures into their own files to be \input.

% For more information on \input and \include, see:
%  http://tex.stackexchange.com/questions/246/when-should-i-use-input-vs-include


% Formatting rules for theses are here: 
%  http://www.ucl.ac.uk/current-students/research_degrees/thesis_formatting
% Binding and submitting guidelines are here:
%  http://www.ucl.ac.uk/current-students/research_degrees/thesis_binding_submission

% This package goes first and foremost, because it checks all 
%  your syntax for mistakes and some old-fashioned LaTeX commands.
% Note that normally you should load your documentclass before 
%  packages, because some packages change behaviour based on
%  your document settings.
% Also, for those confused by the RequirePackage here vs usepackage
%  elsewhere, usepackage cannot be used before the documentclass
%  command, while RequirePackage can. That's the only functional
%  difference as far as I'm aware.
\RequirePackage[l2tabu, orthodox]{nag}

% ------ Main document class specification ------
% The draft option here prevents images being inserted,
%  and adds chunky black bars to boxes that are exceeding 
%  the page width (to show that they are).
% The oneside option can optionally be replaced by twoside if
%  you intend to print double-sided. Note that this is
%  *specifically permitted* by the UCL thesis formatting
%  guidelines.
%
% Valid options in terms of type are:
%  phd
%  mres
%  mphil
%\documentclass[12pt,phd,draft,a4paper,oneside]{ucl_thesis}
\documentclass[12pt,mphil,a4paper,oneside]{ucl_thesis}

% Package configuration:
%  LaTeX uses "packages" to add extra commands and features.
%  There are quite a few useful ones, so we've put them in a 
%   separate file.
\input{MainPackages}
\usepackage{tabularx}

\usepackage{graphicx}
\usepackage{url}
\usepackage{color}
\usepackage[utf8]{inputenc}
\usepackage{amsmath}

\usepackage{tabularx} 
\setlength\extrarowheight{2pt} % make the tables look less cramped
\usepackage{graphicx}
\usepackage{adjustbox}
\newcommand{\ca}{\mbox{C$\alpha$}}
\newcommand{\VH}{\mbox{V\kern-.1667em \lower.5ex\hbox{\scriptsize H}}}
\newcommand{\VL}{\mbox{V\kern-.1667em \lower.5ex\hbox{\scriptsize L}}}
\newcommand{\VHVL}{\mbox{\VH/\VL}}
\newcommand{\CH}[1]{\mbox{C\lower.5ex\hbox{\scriptsize H}#1}}
\newcommand{\CL}{\mbox{C\kern-.0833em \lower.5ex\hbox{\scriptsize L}}}
\newcommand{\FV}{\mbox{\it Fv}}
\newcommand{\Fab}{\mbox{\it Fab}}

\newcommand{\e}[1]{\mbox{$\times 10^{#1}$}}
\newcommand{\etal}{~\emph{et al.}}
\newcommand{\degree}{\mbox{${}^{\circ}$}}
\newcommand{\lilian}[1]{ {\color{red}{\bfseries Lilian:} #1}}
\newcommand{\andrew}[1]{ {\color{green}{\bfseries Andrew:} #1}}
\newcommand{\rewrite}[1]{{\color{blue}{\bfseries Andrew to rewrite:} #1}}

\usepackage{soul}
\newcommand{\highlight}[1]{\hl{#1}}
% \newcommand{\highlight}[1]{{\color{cyan} #1}} % Use this if soul package not available!

\let\shortcite\cite
\emergencystretch 1in


% Sets up links within your document, for e.g. contents page entries
%  and references, and also PDF metadata.
% You should edit this!
\input{LinksAndMetadata}

% And then some settings in separate files.
\input{FloatSettings} % For things like figures and tables
\input{BibSettings}   % For bibliographies

% These control how many number sections your subsections will take
%    e.g. Section 2.3.1.5.6.3
%  and how many of those will get put into the contents pages.
\setcounter{secnumdepth}{3}
\setcounter{tocdepth}{3}


\begin{document}

\nobibliography*
% ^-- This is a dumb trick that works with the bibentry package to let
%  you put bibliography entries whereever you like.
% I used this to put references to papers a chapter's work was 
%  published in at the end of that chapter.
% For more information, see: http://stefaanlippens.net/bibentry

% If you haven't finished making your full BibTex file yet, you
%  might find this useful -- it'll just replace all your
%  citations with little superscript notes.
% Uncomment to use.
%\renewcommand{\cite}[1]{\emph{\textsuperscript{[#1]}}}

% At last, content! Remember filenames are case-sensitive and 
%  *must not* include spaces.
% I may change the way this is done in a future version, 
%  but given that some people needed it, if you need a different degree title 
%  (e.g. Master of Science, Master in Science, Master of Arts, etc)
%  uncomment the following 3 lines and set as appropriate (this *has* to be before \maketitle)
% \makeatletter
% \renewcommand {\@degree@string} {Master of Things}
% \makeatother

\title{Upgrade Report}
\author{Lilian Denzler}
\department{UCL Department of Structural and Molecular Biology}

\maketitle


\begin{Overview} % 300 word limit

The following report summarizes two projects: The T-Cell Receptor Numbering Project, as well as the Qualiloop project. 
In the first instance, a software package was created to reliably number T-Cell Receptor sequences. An accurate numbering package is essential for future proposed project within the PhD. The guiding idea for this section is to re-create a numbering system such as the in-house software used to number antibody sequences, applied to TCRs. 

Secondly, a 3D-model quality predictor of a Complementary Determining Rgion (CDR-H3) is presented. Machine learning techniques are implemented to predict the accuracy of CDR-H3 3D-models generated by antibody modelling software such as abYmod. The predictor is made available at \url{http://www.bioinf.org.uk/abs/qualiloop/}. 

Lastly, future directions for the PhD project are explored. 

\end{Overview}

%\begin{acknowledgements}
%I am hugely grateful to my primary supervisor Prof. Andrew Martin and chair of my committee for the invaluable patience %and feedback. I would also like to thank my secondary supervisor Prof. Adrian Shepherd and my tertiary supervisor Prof. %Benny Chain, who provided their expertise and guidance. Additionally, this endeavor would not have been possible without %the generous support from the MRC, who financed my research. Lastly, I would like to thank Immunocore, the iCASE %collaborating industrial partner. 

%\end{acknowledgements}

\setcounter{tocdepth}{2} 
% Setting this higher means you get contents entries for
%  more minor section headers.

\tableofcontents
%\listoffigures
%\listoftables


%\chapter{Introductory Material}
\label{chapterlabel1}
A robust sequence numbering method for T-cell receptors is important for all work with T-cell receptor sequence data. Universal numbering schemes and correct residue numbering is vital for sequence analysis and comparison. We present a numbering software for T-cell receptor sequences that will reliably implement a set of popular numbering schemes. The software also enables T-cell receptor sequences to be labeled using numbering schemes commonly used for antibodies, which will facilitate studying the differences and commonalities of T-cell receptors and antibodies. It will also enable antibody-based tools to be adapted to T-cell receptor sequences. The numbering software is made available at http://www.bioinf.org.uk/abs/

 

\chapter{T-Cell Receptor Sequence Numbering}
\label{chapterlabel1}
A robust sequence numbering method for T-cell receptors is important for all work with T-cell receptor sequence data. Universal numbering schemes and correct residue numbering is vital for sequence analysis and comparison. We present a numbering software for T-cell receptor sequences that will reliably implement a set of popular numbering schemes. The software also enables T-cell receptor sequences to be labeled using numbering schemes commonly used for antibodies, which will facilitate studying the differences and commonalities of T-cell receptors and antibodies. It will also enable antibody-based tools to be adapted to T-cell receptor sequences. The numbering software is to be made available at \url{http://www.bioinf.org.uk/abs/qualiloop/TCRnum}


Reliable sequence numbering is vital for sequence analysis and comparison. Given the lack of a universally agreed upon numbering scheme to be implemented for T-cell receptor sequences, sequence comparisons can be non-trivial. 
If TCR sequences can also be correctly numbered using schemes commonly implemented for antibody sequences, this will facilitate further research into the likeness of antibody and TCR-characteristics. 
In a paper comparing antibody and TCR CDRs,\cite{Wong2019} it was found that TCR and antibody CDRs occupy distinct areas of structural space. Understanding more about how the two relate may lead to a greater understanding of TCR and their functionality. 
The most commonly used antibody numbering schemes are arguably Kabat-, and Chothia-numbering. Therefore, numbering TCRs using these systems would be of great value for comparing TCR and antibody sequences. IMGT and Aho numbering, as two additional popular numbering schemes,will also be prove useful. 
Furthermore, having correctly numbered TCRs using the same numbering schemes commonly used for antibody sequences will facilitate the re-writing of antibody-geared tools created by Prof. Martin for TCRs. 
There are not many publicly available options available for TCR sequence numbering. An example of already existing software is ANARCI\cite{Dunbar2016}, which is a numbering tool that handles antibody as well as TCR sequences of human or murine origin. TCR sequences may be numbered according to Aho or IMGT schemes.  Furthermore, an unpublished numbering tool can be found on the tcrdb server\cite{Chen2021}, which offers a choice of Kabat or Aho numbering. However, the methods and accuracy of this tool are not publicly available. Therefore, the generation of reliable, robust numbering software that can utilize the Kabat, Chothia, IMGT and Aho scheme, would be very useful for future projects involving TCR sequences. This is especially true, when one considers the fact that all further sequence analyses is subject to correct numbering.

\section{Results}

Firstly, a web interface was created to interactively display different numbering schemes interactively. The different numbering schemes can be selected, as well as the chain type. Insertion/deletion sites according to the selected numbering scheme are denoted, as well as the CDR locations based on Kabat definitions transferred to the TCR context \ref{fig:schemes}. 
\begin{figure}
    \centering
    \includegraphics[width=\textwidth,height=\textheight,keepaspectratio]{schemes-1.png}
    \caption{The different numbering schemes for TCR sequences. The placement of insertions/deletions and position of the complementarity determining regions implied by the numbering scheme are displayed. (Not all numbering schemes are displayed here. For the complete set visit \url{http://www.bioinf.org.uk/abs/qualiloop/TCRnum} }
    \label{fig:schemes}
\end{figure}

In order to obtain a reliable numbering programme, a set of correctly numbered sequences for each of the numbering schemes is needed. These sequences are sorted by chain type and organism and are stored in MongoDB database collections along with the correct numbering \ref{fig:database_creation} Once the database is set up, the CDR regions within the different numbering schemes are defined. As no official Kabat or Chothia definitions of the CDR regions exist, these were arrived at using an alignment with the equivalent antibody numbering scheme\cite{AAAAA}.
Furthermore, by analysing the sequences in the newly produced mongo database, a Kabat-style definition for the TCR can be proposed \ref{fig:Aho_logo}, \ref{fig:IMGT_logo}.

\begin{figure}
    \centering
    \includegraphics[width=\textwidth,height=\textheight,keepaspectratio]{database_creation.png}
    \caption{Consensus Sequence Creation. First, a MongoDB database is created, with three collections. These contain the parsed sequences with their meta data of Kabat, IMGT and Aho pre-numbered sequence databases. The sequences are then clustered (categorized by chain type). Residue frequency is analysed among the clusters to yield multiple consensus sequences, as well as conserved sequences.}
    \label{fig:database_creation}
\end{figure}


\begin{figure}
    \centering
    \includegraphics[width=\textwidth,height=\textheight,keepaspectratio]{Aho_logo.png}
    \caption{ Sequence Logo of TCR sequences in tcr3d database (Aho-numbered). Few conserved residues can be seen, as well as areas of greater sequence variability. Although some interesting features may be seen within this logo, there are many partial sequences in the  tcr3d database,which may distort some regions. }
    \label{fig:Aho_logo}
\end{figure}

\begin{figure}
    \centering
    \includegraphics[width=\textwidth,height=\textheight,keepaspectratio]{IMGT_logo.png}
    \caption{Sequence Logo of TCR sequences in STRCdb database (IMGT-numbered). Few conserved residues can be seen, as well as areas of greater sequence variability. A similarity with Figure 2 can be seen. (Note:  the two figures have not been aligned according to sequence).}
    \label{fig:IMGT_logo}
\end{figure}

A consensus sequence can also be obtained for these sequences. Through prior sequence clustering by sequence identity, more informative consensus sequences can be produced\ref{fig:database_creation} The best-matching consensus sequence will then be selected for each sequence that is to be numbered in further steps. 

\subsection{Anchor Sequence Alignment}
In this equivalent to the AbNum-algorithm, sequence propensity profiles which are aligned to the sequence to find the CDR boundaries are used. By aligning short anchor sequences (10 \AA) built from sequence propensity profiles before and after CDR regions, the numbering can be filled in. The amino acid numbers are filled in from both sides in turn.

The sequence propensity profiles were built for each of the framework regions and CDRs using all organisms and chain types. Then, profiles were build for FR and CDRs separated by chain type, then separated by organism and lastly by both organism and chain type. 

[Most reliably, a large set of profiles separated by both organism and chain type will be used. The best match is selected for the first anchor alignment. The chain type and organism selection is then verified by matching other anchors of the same group to the sequence.]

Furthermore, the template was moved back by 5 residues in CDR3, due to a maximum deletion of 5AA in the V-region, to improve sequence profile alignment accuracy.


For the Kabat sequencing scheme, the profiles were built using exclusively sequences from the Kabat database. The profiles were used to number the entire cleaned Kabat dataset. For sequences that were incorrectly numbered, these were removed and new sequence profiles were built iteratively.
The manually removed sequences (after manually checking these were correctly numbered) were then used to create separate sequence profiles. 

As a fall-back a full length consensus sequence is used. The best consensus is determined by iterating through a set of consensus sequences, which are yielded from sequence clusters. 

\subsection{Modified Needleman-Wusch Algorithm}
This is a dynamic approach that implements a modification of the Needleman-Wunsch algorithm. 
Similar to AbRSA\cite{Li2019}, an algorithm for robust antibody numbering, a differentiation between CDR, 
FR and insertion positions within the scoring system is introduced. This yields a more refined alignment. 
Parameter optimization was employed for best results.

Multiple consensus sequences are obtained each for TCR chains α,β,γ,δ. The best-fitting consensus sequence for the input sequence is chosen, matching the input chain-type. Using the different CDR-definitions and insertion positions according to the numbering schemes, the consensus sequence residues are categorized as belonging to 1) framework region, 2) CDR, 3) insertion positions, 4) conserved positions. A score is calculated according to



\noindent With cells in the array numbered from 1:
\begin{equation*}
s[i,j] = (S(a[x,i],a[y,j]) *q) + \max\left\{ \begin{array}{l}
       s[i-1, j-1]\\
       s[i-1, J = (j-2)\ldots 1] - g\\
       s[I = (i-2)\ldots 1, j-1] - g
       \end{array} \right.
\end{equation*}
\noindent where:
\begin{equation*}
  \begin{array}{ll}
    S[a,b] &= \text{BLOSUM62 scoring matrix score for amino acids } a \text{ and } b\\
    a[x,i] &= \text{the amino acid at position } i \text{ in sequence } x \\
    g      &= \left\{ \begin{array}{l}
        P_{CPs}  \ if j \in conserved positions\\
        P_{IPs} \  if j \in insertion positions\\
        P_{FRs} \  if j \in Framework positions\\
        P_{CDRs} \  if j \in CDR positions\\
       \end{array} \right. \\
       
     q      &= \left\{ \begin{array}{l}
        S_{CPs}  \ if j \in conserved positions\\
        1 \  if j \in others\\
        \end{array} \right. \\
        
    n      &= (j-1) - J \text{ -or- } (i-1) - I\\
  \end{array}
\end{equation*}

\clearpage

The  values of P\textsubscript{CPs},  P\textsubscript{IPs} , P\textsubscript{FRs} and P\textsubscript{CDRs} are gap penalties. The value of S\textsubscript{CPs} is a weight for a matched conserved residue. The values are defined in the AbRSA paper \cite{Li2019}.

\subsection{HMM alignment}
In this approach Hidden Markov Model profiles are first generated for short sequences of 5 amino acids prior to and after a CDR-boundary. By aligning these to the sequence correctly, the start and end of these regions can be determined. After the CDR positions are correctly determined and verified, the numbering can be filled in according to the respective numbering scheme. 

\section{Methods}

\subsection{Database Creation}
For Kabat sequencing, the Kabat TCR database was downloaded. The sequences are separated by chain and organism. There is no need for sequence alignment using this database, as the input is pre-aligned. The sequences are then uploaded into a database (MongoDB). This is done to build a comprehensive database of manually annotated TCR sequences of all different numbering schemes investigated, which was not currently available. IMGT-annotated sequences are taken from the StCRDab\cite{Leem2018} database, which contains PDB files of human and mouse TCRs (αβ and γδ), Aho-numbered sequences were taken from the tcr3d database5\cite{Gowthaman2019} 

\subsection{Clustering Methods}
\subsubsection{CD-HIT}
Sequence clustering with CD-HIT\cite{Li2006} (70\% sequence identity)

\subsection{Consensus Sequence}
 Consensus sequences built by selecting residues with a position specific score (PSS) of above 50\%. Conserved residues are classified as those with a PSS \> 95\%.
 

\section{Discussion}

Upon evaluation of all tested numbering methods, the most robust approach was determined to be anchor sequence alignment. The numbering, when tested on the cleaned Kabat dataset, yielded accurate numbering in all but 12 cases. Of the failed instances, 4 of these sequences were sourced from "other" organisms. The source of the sequences could not yet be determined. 
The short-coming of the AbRSA-based method may be explained by insufficient optimization for the values of the gap penalty values and conserved position weights. 
A rudimentary version of the anchor-based method will be made available for sequence numbering at  \url{http://www.bioinf.org.uk/abs/qualiloop/TCRnum}. 



\include{Chapter2}
%\chapter{Future Directions}
\label{chapterlabel3}

\section{Make Stability Predictions}
1. Make stability predictions
-  Retrain antibody Tm-predictor on TCRs
-  Use melting-point library at immunocore
-  Sequence vs. structure based predictions
\section{Make Modifications}
2. Make modifications
-  Analyse packing residues
-  Compare to antibody packing residues
-  Reengineer to stabilize

\section{TCR Modelling}
extract features or structural rules to aid modelling
Alphafold testing
-  Domain Packing Prediction
-  Affinity enhancement
-  Tool for flagging unusual patches on surface
-  Deamidation prediction
Prevent degradation, enhance stability

\subsection{Martin-like Numbering}
Own numbering scheme, based on Kabat but with insertions and deletions at the structurally correct positions (Martin numbering)


%\include{Appendices} 
% You could separate these out into different files if you have
%  particularly large appendices.

% This line manually adds the Bibliography to the table of contents.
% The fact that \include is the last thing before this ensures that it
% is on a clear page, and adding it like this means that it doesn't
% get a chapter or appendix number.
\addcontentsline{toc}{chapter}{Bibliography}

% Actually generates your bibliography.
\bibliography{example}

% All done. \o/
\end{document}
